\documentclass{sig-alternate}

\usepackage[usenames,dvipsnames]{color}
\usepackage{xspace}

\usepackage{fancyvrb}
\DefineVerbatimEnvironment{code}{Verbatim}{fontsize=\small}
\DefineVerbatimEnvironment{example}{Verbatim}{fontsize=\small}

% Looks better (and more concise) than Times New Roman
\usepackage{times}
% Better spacing
\usepackage{microtype}
\usepackage[normalem]{ulem}
\usepackage{enumitem}
% Caption package both lets you set the spacing between figure and caption
% and also makes the \figref{} point to the right place.
\usepackage[font=bf,aboveskip=0pt,belowskip=-12pt]{caption}
% Get the caption package to work with ACM style
\DeclareCaptionType[placement=b,within=none]{copyrightbox}
%%%%%%%%%%%%%%%%%%%%%%%%%%%%%%%%%%%%%%%%%%%%%%%%%%%%%%%%%%%%%%%%%%%%%
\newfont{\mycrnotice}{ptmr8t at 7pt}
\newfont{\myconfname}{ptmri8t at 7pt}
\let\crnotice\mycrnotice%
\let\confname\myconfname%

\permission{Permission to make digital or hard copies of all or part of this work for personal or classroom use is granted without fee provided that copies are not made or distributed for profit or commercial advantage and that copies bear this notice and the full citation on the first page. Copyrights for components of this work owned by others than the author(s) must be honored. Abstracting with credit is permitted. To copy otherwise, or republish, to post on servers or to redistribute to lists, requires prior specific permission and/or a fee. Request permissions from Permissions@acm.org.}
\conferenceinfo{}{DMAH 2015, September 4th, Kohala Coast, USA}
\copyrightetc{}
\crdata{}

% Allow comments by name
\newcommand{\note}[2]{
    \textbf{\textcolor{#1}{#2}}
}
\newcommand{\comment}[1]{\note{red}{#1}}
\newcommand{\dave}[1]{\note{PineGreen}{Dave: #1}}
\newcommand{\magda}[1]{\note{Cyan}{Magda: #1}}
\newcommand{\tom}[1]{\note{OrangeRed}{Tom: #1}}

% To remove all notes, uncomment the next line.
%\renewcommand{\note}[2]{\unskip}

\newcommand{\ie}{{\em i.e.}\ }
\newcommand{\eg}{{\em e.g.}\ }
\newcommand{\ea}{{\em et al.}\xspace}
\newcommand{\aka}{{\em a.k.a.}\ }

\usepackage{url}
\usepackage[pdftex]{hyperref}

% Figures
\newcommand{\figureimgs}{
\begin{figure}[tbp]
  \centering
  \includegraphics[width=\linewidth]{figures/x_ray_image_montage}
  \vspace{-12pt}
  \caption{\label{fig:imgs}
  	X-ray diffraction images.
    A) An X-ray diffraction image exhibits $d_{10}$, $d_{11}$, and
    $d_{20}$ peaks; the first and last are proportional to the
    distance between adjacent rows of thick filaments. The $d_{10}$ is
    the primary peak being targeted for automated detection and
    fitting. B) In this image, two fibers at slight angles to each
    other have generated multiple $d_{10}$ peaks which must be
    segregated during the detection process.  
	}
	\vspace{-2pt}
\end{figure}
}

\newcommand{\figureworkflow}{
\begin{figure}[tbp]
  \centering
  \includegraphics[width=\linewidth]{figures/img_analysis}
  \vspace{-8pt}
  \caption{\label{fig:workflow}
  	Analysis workflow.
    The blocked center region is detected and used for registration.
    Low signal regions and edges are masked. The image is smoothed and
    peaks are detected, clustered, and classified. The background
    (minus that surrounding diffraction lines) is computed and
    subtracted. Regions of interest surrounding the $d_{10}$ peaks are
    extracted and the generating distributions are fit with an MCMC
    sampler. We marginalize across peak parameters other than those of
    interest.  
	}
	\vspace{-2pt}
\end{figure}
}

\begin{document}
\title{Automated Analysis of Muscle X-ray Diffraction \\ Imaging with MCMC}
\author{C.\ David Williams$^{1,2}$,  Magdalena Balazinska$^2$, Thomas Daniel$^1$ \\
\affaddr{$^1$Department of Biology, $^2$Department of Computer Science \& Engineering}\\
\affaddr{University of Washington} \\
}
\maketitle



% \dave{Presumably this gets filled in later?}
%    \vspace{-8pt}
%    \category{H.2.4}{Database Management}{Systems---Distributed databases, Query processing, Relational databases}
%    \vspace{-8pt}
%    \keywords{Myria; parallel data management; Cloud service; astronomy}


\sloppy

%%%%%%%%%%%
%\section{Questions addressed}
%\label{sec:addressed}


%In this work, we focus on the following questions:

%\begin{itemize}[noitemsep]
%\item Are sufficient conserved markers available in small-angle scattering diffraction images for automatic segmentation?
%\item Does MCMC distribution fitting provide reliable estimates of diffraction peak properties such as center and decay?
%\end{itemize}


%%%%%%%%%%%
\section{Motivation}
\label{sec:motivation}


Biologists have studied muscle structure and dynamics since antiquity
but it is only with the advent of X-ray diffraction techniques in the
1960s that it became possible to visualize the orientation and spacing
of the semicrystalline proteins that generate and transmit force
within muscle \cite{Millman1998}. In X-ray diffraction, structural
information is recorded when an X-ray beam passes through and scatters
off a muscle sample to produce an image such as that shown in
Figure~\ref{fig:imgs}.

Today, human experts manually extract structural parameters
from X-ray images using NIH ImageJ. This produces repeatable
measurements but remains subjective, fails to provide confidence
intervals for the measured values, and is not reproducible by naive
digitizers. Additionally hand digitization is a time-intensive
analytic technique, with a single digitizer able to process only a few
hundred images a day.  This has historically been sufficient, but new
high-speed imaging systems are allowing the investigation of
short-timescale components of muscle contraction and generating data
sets with many thousands of images. The need for an automated and
reproducible image analysis tool chain is clear. 

The goal of our project is to build a service for the automated
analysis of X-ray images of muscle structures.  Users should specify
the analysis that they need using a declarative query interface and
the system should automatically process the user's image database.  In
this extended abstract, we present the first components of the
processing tool chain at the heart of this service. The toolchain
first segments the diffraction image into regions of interest using
conserved features and then samples the possible parameter values with
a Markov chain Monte Carlo approach.


%%%%%%%%%%%
\section{Prototype}
\label{sec:proto}


We initially focus on measuring the $d_{10}$ parameter, a crucial
spacing in muscle shown in Figure~\ref{fig:imgs}. The $d_{10}$ spacing
determines the distance which muscle's molecular motors must bridge in
order to bind and generate force \cite{Williams2013}. This distance
changes during contraction, regulating the force produced.

Images generated during experiments share several key features, which
serve as challenges or fiduciary marks during analysis. As seen in
Figure~\ref{fig:imgs}, the brightest part of the image background is
occluded by a circular stop. This physical block is put in place to
prevent damage to the detector from the high photon flux seen at the
center of the X-ray beam. Surrounding the blocked region, the
remainder of the image displays an exponentially decaying background.
Interrupting this background are symmetric pairs of diffraction peaks
we locate and model.  Our core data analysis pipeline includes two
steps: image segmentation and image modeling with MCMC processes.

\figureimgs



\subsection{Image segmentation}

The dark central circular blocked region is first located and acts as
a relative reference point for subsequent operations. Consistent with
experimental design, it is assumed to contain the center of the
diffraction pattern, although the center of the diffraction pattern is
not assumed to be at the center of the blocked region. The edge of the
background surrounding the block is the brightest region, so the image
is first split between areas with values less than and greater than
two standard deviations above the mean. This partitioning yields a
binary image where the center blocked region is surrounded by a halo
of the upper end of the pattern background and occasional dots where
diffraction peaks rise more than two standard deviations above
background. This binary image is converted to a hierarchical contour
set with OpenCV. The blocked region is taken as the inner-most contour
and then modeled as the smallest enclosing circle, shown as dark
magenta in Figure~\ref{fig:workflow}. 


\dave{I cut this down and simplified. It can be removed or further
reworked if desired.}
With the central blocked region located, we identify local maxima in 3
by 3 pixel groups after Gaussian smoothing and mask those found in
regions unlikely to provide peaks of interest.  Resulting maxima are
accepted or rejected as peaks based on masking parameters.  Masked
rejection-regions consist of: 1) a circular zone around the central
blocked region where the blocking generates non-peak local maxima, 2)
areas below the 80th percentile where detector noise dominates, 3)
areas near the image edge where peaks are partially cropped. The
resulting unmasked area from which we keep maxima is depicted as light
purple overlay in Figure~\ref{fig:workflow}.

\figureworkflow

Next, maxima are clustered into peak pairs based on their distance
and angle from the center of the blocked region. Starting with those
maxima nearest the blocked region, a corresponding maxima is sought an
equal distance away from the blocked region and located such that the
angle formed by the two maxima and the center of the blocked region is
180$^\circ$. If no matched maxima exists within a 10\% tolerance, the
maxima is discarded as a spurious peak. With peak pairs now identified
(shown as color matched dots in Figure~\ref{fig:workflow}), the
diffraction center is identified by taking the mean location between
peak pairs and the background is fit and subtracted. 

The background is subtracted by first masking arcs encompassing peak
pairs and then fitting a double exponential to a radial profile of the
remaining image. An arc swept 12$^\circ$ out on either side of each
peak pair masks the effect of the diffraction peaks on the background
(shown as a light gray arc under the peak pairs in
Figure~\ref{fig:workflow}). We calculate a radial profile of the image
around the pattern center, omitting diffraction line regions. We fit a
double exponential function of the form $background = a+ b e^{-x c} +
d e^{-x e}$ to the radial profile. From these parameters we generate
an estimated background image and subtract it from the real image,
allowing us to extract the $d{10}$ peaks as regions of interest (ROIs)
unhindered by an overlaid diffraction background.

\subsection{Image modeling with MCMC processes}

With the background subtracted and the $d_{10}$ peaks identified and
isolated from the rest of the image as ROIs, we apply Markov chain
Monte Carlo (MCMC) sampling to determine the probability distributions
from which the peak parameters could be drawn. We treat the peaks as
being drawn from an underlying Pearson VII distribution, commonly used
to fit X-ray diffraction peaks. This process allows us to generate
possible peak matches using five parameters: peak center x-location,
peak center y-location, peak height, peak spread, and peak decay rate.
We perform an initial peak fitting by residual minimization between a
generated peak and the extracted ROI. This gives us a set of starting
parameters that we use, with random variation, to initialize the
positions of the MCMC agents that will explore the model space. 

Before MCMC sampling we must define our query's likelihood and prior.
We choose a flat prior as our initial information about the model is
minimal. To calculate the likelihood we represent each pixel's photon
count as a Poisson process in the form $P(d|m)=e^{-m}
\left(m^{d}/d!\right)$ where $m$ is the model value and $d$ is the
experimental data value. These functions are fed into \textit{emcee},
an efficient MCMC analysis Python library \cite{ForemanMackey2013}.
After a burn in period of 100 steps, the sampler histories are erased
and a further 1000 steps are run to generate the posterior probability
distributions of our peak parameters. 

One of MCMC modeling's convenient features is that extracting only a
subset of parameters marginalizes across those we discard. That is,
when we are interested in only the x- and y-locations of the peak
center to precisely calculate $d_{10}$ spacing (as in
Figure~\ref{fig:workflow}), we automatically integrate our uncertainty
about peak height, spread, and decay.


%%%%%%%%%%%
\section{Preliminary Evaluation}
\label{sec:eval}

We apply our workflow to a test corpus of 1,220 images generated using
X-ray diffraction during insect flight muscle research at Argonne
National Laboratory's BioCAT Beamline. Sample high-quality and
challenging images are shown in Figures~\ref{fig:imgs}A and
\ref{fig:imgs}B. Our technique segments these images into regions of
interest, identifies the approximate locations of the diffraction
peaks, subtracts background from the regions surrounding the peaks,
and computes the distribution of possible parameters which underly the
shown peaks.  We find that our initial image segmentation step
successfully models the center blocked region of each image in greater
than 99\% of images in our test corpus of 1,220 images. The overall
process allows us to calculate peak-to-peak distances to sub-pixel
accuracy with a confidence interval of 90\%. MCMC sampling combined
with image segmentation allows us to precisely, accurately, and
automatically locate the $d_{10}$ peak center and thus calculate the
lattice spacing measured by a diffraction image to within 0.03 nm.

Because the images in our corpus are a standard sample of those
produced by high-speed X-ray diffraction, our positive preliminary
results are a strong indication for the potential of this approach.


%%%%%%%%%%%
\section{Challenges and Next Steps}
\label{sec:challenges}

Our initial data processing pipeline produces an automated,
reproducible, objective estimate of relevant image parameters but the
following challenges remain:

\begin{itemize}[noitemsep]
\item Development of a declarative language to describe processing
    steps will speed use of this technique and ease reproducibility.
    The key question is to define the types of operations that users
    should be able to specify and how to specify them. Our goal is to
    generalize to a broad set of analysis needs for X-ray images of
    muscle structure.
\item Packaging of this tool chain into a cloud deployable
    containerized image will enable trivial scaling to work with
    larger datasets. Furthermore, the ability to access to tool chain
    directly through a web browser with automatic back-end deployment
    of the analysis pipeline will facilitate adoption. 
\item Application of these techniques to coming ultra-high
    temporal-resolution images with far lower signal:noise will strain
    autosegmentation and peak-fitting techniques. 
\end{itemize}


%%%%%%%%%%%
%\section{Acknowledgments}
%\label{sec:acknowledgments}
%
%This work was supported in part by the Army Research Office through
%ARO Grants W911NF-13-1-0435 and W911NF-14-1-0396, an award from 
%the Gordon and Betty Moore Foundation and the Alfred P Sloan 
%Foundation, the Washington Research Foundation Fund for Innovation 
%in Data-Intensive Discovery, and the UW eScience Institute.
%
%We thank Jake VanderPlas for helpful discussions of statistical
%techniques, Tom Irving for advice on X-ray imaging, and Simon Sponberg
%for the sharing of diffraction images. 


\scriptsize
\bibliography{articles,paper}
\bibliographystyle{abbrv}

\end{document}
